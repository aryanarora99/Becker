\documentclass{article}
\usepackage[utf8]{inputenc}
\documentclass[12pt]{article}
\linespread{2}
\usepackage{times}

\usepackage{pgfplots}
\pgfplotsset{compat = newest}
\usetikzlibrary{positioning, arrows.meta}
\usepgfplotslibrary{fillbetween}
\usepackage{amsmath}
\usepackage[margin=1 in]{geometry}

\title{Economics of The Family Final Paper}
\author{Aryan Arora}
\date{February 2022}

\begin{document}
\maketitle

In \textit{A Treatise on the Family} Becker conducts an economic analysis on the consumption decisions that individuals in a family make. Becker examines the investments individual make into market and household human capital, the trade-offs parents face when deciding how many children to have, and the choices people make when entering the marriage market. In this paper I examine Becker's arguments in 3 components. First, I examine the merits of  using economics in 'non-market activities.' Next, I examine how successfully those tools can be applied to the decisions that families make. Finally, I examine how we go beyond Becker and work off his framework. I argue that economics is applicable to 'non-market activities', and is applicable in examining families to a limited extent.

\subsection{Application of Economics to Non-market Activities:}

In Charles Shultz's \textit{National Output and Income}, published years before Becker’s \textit{A Treatise on the Family}, the author argues that there are market and non-market activities. Market activities are those which deal with labor participation and the goods and services people purchase to fulfill their needs and desires. Non-market activities are anything else—watching TV, tending to the garden, child-rearing, etc.. The author goes on to say that the applications of Economics are limited to only market activities. Becker flips this argument on its head and says that anything can be examined economically. This is the foundation of Becker's economic analysis of the family. Thus any merits we assign to his arguments on the Economics of families must be preceded by an examination on the merits of this argument. 

Becker's argument is particularly appealing because he frames all activities as inputs into certain commodities—health, fulfillment, etc.. This argument expands economic thinking in all domains to a dimension previously unrecognized. Even if we limit ourselves to Shultz’s argument and define market activities as the limits of Economics, we are left questioning why consumers make the consumption decisions they make; Why some participants in the labor force are willing to supply greater quantities of time to their jobs than other workers with the same jobs and monthly expenditures, or why some consumers purchase more of a good than their ‘need.’ We can only broadly chalk these economic choices up to the tastes of the consumers without any deeper investigation. 

Becker’s framing allows us to more conclusively examine these decisions. We can examine why one worker might choose to work more because their work, unlike for other workers, is a more valuable input to the commodity of fulfillment or self confidence. Becker’s framing allows us to create a formula (a unique combination of inputs) for each individuals' tastes—holding the final commodity produced constant. In making a decision about which viewpoint economists ought to adopt, we can use economic tools to examine each approach. Becker’s approach, without limiting the use of Economics at the time or constricting the power of Economists, expanded the domains Economics could examine and the depth to which consumption choices could be examined. Simply put, Becker’s viewpoint was a Pareto improvement over the status quo at the time. 

Becker’s framing also allows us to tell a more complete complete story of the economic decisions that individuals make. As they are offered new inputs to the same commodities, their consumption choices may change. This does not mean that their preferences have changed, but rather suggests that they are—just as every economist would hope they would be—rational consumers who will seek to produce commodities at least cost. For instance, an individual who has children may begin to work less than they did prior to having children, even if the additional help is not needed at home. This would be perplexing to the [original author] school of thought, but using Becker’s framework we can easily quantify this. What the individual cares about is not their income or their ability to consume xyz goods, it is their final commodity. Lets assume in this case that is fulfillment. Because they can achieve fulfillment much more quickly by being at home—by watching their child’s first steps, their small triumphs, and growth—the individual will chose to work less. Their taste for work has not changed—they still get the same utility from work as before—but they are now able to, on the margin, produce a substitute input to their fulfillment commodity at lower cost.

This effect can be extended to a substitution between monetary consumption and time so that the parent may no longer choose to work as much because spending an extra hour with their child provides more utility that a new pair of shoes (assuming that 1 hour of labor is equivalent to a new pair of shoes for them). In this case the individual will continue being with their child  until the point of indifference: $\theta_{child} = \theta{working} + \theta{product of work}$ where $\theta$ is aggregate contribution to all commodities. The use of "aggregate contribution" here refers to the fact that nothing is an input to just one commodity so even if one consumption choice is a greater input to the fulfillment commodity, the consumer will chose the consumption choice which makes the greatest contribution to all commodities (or weighted based on their preferences for commodities).  

This is the remarkable point that Becker makes but never states: that the true constraint of consumption is time. While it is true that those who are well-endowed or work high paying jobs have some efficiency coefficient that allows them to consume more, their ability to fulfill their commodities is limited by the same constraint as those that are poor: time, and they too face the same trade-offs between the various inputs to those commodities. 

\subsection{Application of Economics to Non-market Activities:}

One of the primary applications of Becker's expansion of Economics is an examination of the labor decisions of families. Becker's analysis in this context is as valuable as it is limited. Becker is able to aptly examine the incentives offered to families, but much of his argumentation is limited in one capacity or another. 

Perhaps the strongest of Becker's arguments is the argument for investments into human capital for individual members of a household. Becker argues that a household is most efficient when there is at least one person who specializes in market activities and one person who specializes only in household activities. Becker suggests that women often specialize in household activities not just because they have a biological advantage in child rearing, but also because considerably more investment is made into their household human capital. His point, seemingly well intentioned in its attempts to emphasize the additional investments needed to make women equally competitive in the marketplace, is rife with lacunas. Setting aside any notion of it being tone-deaf or sexist—both inconsequential to our economic analysis and partially attributable to changing times—Becker's argument feels incomplete. Primarily because he attributes the growth of investments into women's market human capital to rising divorce rates and later ages of marriage. Furthermore because He offers no ambiguity in stating that investments into women's human capital serves as investments into their success in the marriage market. 

The primary challenge I offer with Becker's argument is to his treatment of human and household human capital as being equal. They are not. Unlike with market human capital, we can only value household human capital in terms of market human capital. The ability to keep a house clean is only worth something if there is someone with market capital to pay for a house. To feed or clothe children is only worth something should there be someone with market human capital to provide food and clothing. Becker's argument suggests that women should make large investments into human capital that is only worth something if on decision day a man finds her to be virtuous enough for marriage. The second concern with this line of logic is that household human capital is not as valuable to market human capital because it does not require as large of investments. We know this to be empirically true because parents, independent of their child's sex, fear sending their child to school or college under-prepared in market human capital—it funds a tutoring businesses all across the US—yet the feel much less fear about sending their kids off without knowledge of how to cook, keep their room tidy, or take care of children. These skills can be learned in college and beyond. If not through osmosis of living in a clean house throughout their childhood, children can learn these skills through experiential learning: a filthy dorm. 

My challenge of Becker's argument is not against his fundamental conceptions—it is against the value of his conclusions. I agree with him that there is household and human capital, I agree with him that there is necessary investments that need to be made for both, I agree with him that an 'efficient' household would have one individual specialize in each so that everyone focuses on the domain where they have a comparative advantage. But after reading Becker's argument I was left questioning what the value in efficiency here is—whether anyone really cares about running the most efficient household when deciding how they and their spouse should allocate their time. Families aren't efficient, and that's where Economics comes up short. We can adopt Becker's discourse about how our family is in our utility function but that doesn't adequately explain the behavior of members of a family—Economics isn't enough to tell that story.

Becker attempts to make the case that this efficiency allows families to maximize their production of commodities they care about because everyone is specializing but I remain unconvinced of this argument. First, Becker's treatment of a household all contributing to the 'bucket' of commodities which they all reap from feels empirically untrue. While their fates are tied together, they ultimately produce individual commodities. For instance, a woman who derives fulfillment from her job cannot have the same level of fulfillment if she spends all her time at home while her husband works. His ability to bring back more of the monetary inputs to their fulfilment commodity is not a substitute for her experiential inputs. Furthermore, Becker's own framing indicates that this division of labor does not allow both partners to contribute equally to the production of their joint commodities±—while the labor participant produces money through their labor that can be used for purchasing goods and services that will contribute to their commodities, the person in charge of managing their house only improves the efficiency with which they can covert that input into commodities: they do not produce inputs to the commodities themselves. My final complaint about this line of thinking is what I stated earlier: that the only way the individual who runs the household is able to contribute to their own commodities is through support from the labor participant—if they choose one day to no longer support their spouse, the individual with only human capital and experience in the household will struggle to produce the commodities they wish to. That is not true if the household manager were to choose to no longer support the labor force participant. My argument is not that they won't both struggle post divorce, it is that the individual with the comparative advantage in the household will struggle more. Households moving away from Becker's conception of an 'efficient' household towards one where both partners have large endowments into market capital is about more that shifting marriage and divorce rates—the story that Becker puts forth is simply incomplete, at least by today's standards.

Becker proceeds to make similar arguments throughout the book—arguments that have correctly sorted out the economic incentives or what would be efficient from an economic standpoint, but consistently fail to tell the full story. It reappears when Becker talks about the quality, quantity trade-off when having more children. Here Becker makes an intuitive argument that having more children means having 'lower quality' children. But solely on the metric of how much money their children will make. Becker's argument that lower quality means lower income, and since the child's success is a part of their parents utility function means lower utility for the parent, is reductive. If the intention was solely financial, children are not the most efficient investment. Lets look at children as bond where the payoff is both their financial gains and their contributions to the parents' utility function. Not only is the payoff of children less than most any other bond, but you have to keep investing into the bond for 18 years. Worse yet, even if you are in it just for the contribution to your utility function, children are the riskiest bond. They can actively work against your utility function despite the rotten kid theorem. Here again, because we cannot appropriately model the behavior of children using economics, I found the Becker's conclusion larger unsatisfactory—incomplete.

My point is not to say that Becker's analysis is non-valuable or that his conclusions are incorrect. He is correct in his assessment of a quantity/quality trade-off and in his idea that an efficient household has one person specialize in just one domain because they have a comparative advantage—but often in cases of the family, efficiency is not the desired outcome. It is inefficient in a household to have a wife who wants to commit herself to market activities dedicate all of her time to household activities because she holds a comparative advantage there. Such an outcome would decrease her capacity to the point where it is inefficient for the household (and it is unlikely that there continues to be a household in such scenarios).

One domain where I did find Becker to have a valuable offering was his examination of total investment into inter-generational wealth. Particularly his examination of credit constraints and optimal levels of human capital. In the context of this argument, I am willing to uphold Becker's argument that as a parent your utility is tied to your child's income (independent of what other quantifiers of a 'good life' a parent may have for their children) because their dependence on your non-human capital investments into them not only reduce your total utility, but furthermore take away resources from you that could be used to increase your utility (a lose-lose). Here Becker makes what I think is a revealing point that the optimal level of investment into a child for a credit constrained family is the point where you are willing to invest into their human capital, but are unwilling to leave them any bequests or trust funds. The reason I am more willing to accept this economic argument that those I attempted to refute above is twofold. The first is that when determining how to allocate their resources towards their children, families do tend to think more about making an economically valuable decision as compared to when deciding whether or not to have another child. The second is that in this scenario, the parent is aligning their economic interests with that of their child as is shown through the rotten kid theorem.

\subsection{Moving Past Becker:}

The simplest way to move beyond a text like Becker's is to add pieces to it—to examine what variables Becker ought to have added in his household utility functions, or to debate whether he overstates the importance of certain variables. I won't do either on two accounts. The first is that these are cheap uses of this space—I do not have the technical knowledge to adequately challenge Becker, but much more importantly in attacking the pieces of the argument you are oxymoronically ignoring that you have accepted his arguments and are attempting to improve them. I admit, despite my criticisms of Becker's applications, he has won his arguments with me. I challenge only whether his ideas have been implemented as well as they can—not if they have any underlying validity. Thus I will focus on what we ought to do with Becker's ideas. There are certainly more viable ideas than I could discuss here—but I will suggest a couple possibilities. The first is what I will refer to as the Ikea method, and the second is the inclusion of interdisciplinary examination. 

The Ikea method is taking Becker's ideas, much like you would furniture at Ikea. Now that we have seen the showroom, and we like the pieces he is offering, we can take disassembled pieces, make any necessary modifications, and suit them for our particular uses. For instance, we can Becker's framing that people consume in order to fulfill commodities, that there are joint commodities in households, and that people will seek to best fulfill all of those commodities and expand that beyond just families. If we take this framing we can examine why there might be higher levels of altruism in low income and immigrant communities by stipulating that people will continue to expand the contributions they make to others' utility functions (and the contributions to their utility function by others) until the point where their commodities are fulfilled. Low income and immigrant communities who typically do not do 'prideful' work, may be more willing to contribute to the education costs of the kid down the street or to help out at the local rec center because expanding beyond their own family is necessary to fulfill that utility function.

The interdisciplinary approach to Becker's postulations is to continue what he did—expanding the depth to which we as Economists go to tell the story of consumption choices. This means including sociological and psychological analyses to our Economic examinations of the household and beyond. If we are to accept my argument that Becker tells an incomplete story of the household then we ought to pay little heed to Becker's component of the story, and pay more attention to the missing pieces. For instance, psychology can probably tell us what leads to a household with the most homogeneous production of commodities—where both partners are able to reap from the same commodities instead of fulfilling each's commodity individually. That would help us move closer toward Becker's conception of an efficient family. 

\end{document}
